\documentclass[12pt]{article}

\input{~/Desktop/LosDelDGIIM.github.io/subjects/_assets/preambulo}

\author{Joaquín Avilés de la Fuente \and Arturo Olivares Martos}
\date{\today}
\title{Memoria Práctica \texttt{Lex}\\Modelos de Computación}

\begin{document}    

    \maketitle
    \tableofcontents

    \begin{abstract}
        En la presente memoria se detallará la práctica llevada a cabo en la asignatura de Modelos de Computación del tercer curso del Doble Grado en Ingeniería Informática y Matemáticas de la Universidad de Granada.
        En ella, hemos implementado un menú con diversas funciones para analizar un texto introducido por el usuario, el cual detallaremos más adelante en las distintas secciones de la memoria.
    \end{abstract}

    \newpage
    \section{Introducción}

    En nuestro caso, para no limitarnos a un único caso de uso, lo cual probablemente implicaría usar menos expresiones regulares, hemos decidido implementar un menú con diversas opciones que permiten al usuario analizar un texto introducido por él mismo. Esto proporciona así más versatilidad, a la vez que nos facilita la implementación de un mayor número de expresiones regulares.

\end{document}
